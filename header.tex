
%!TEX encoding = UTF-8 Unicode

%---------------------------------------------------
%
% 図表に関する指定
%
%---------------------------------------------------

%% 表の大きさの自動調整
\usepackage{adjustbox}

%% 欧文の表形式の実現
\usepackage{booktabs}

%% リンクを埋め込んである文字列の強調
\hypersetup{
    colorlinks = true,
    citecolor = [RGB]{24,127,196},
    linkcolor = [RGB]{24, 127,196},
    urlcolor = [RGB]{24,127,196}%orange
}

%% 画像の位置の設置
\usepackage{float} % Don't delete after 2020.06.08 update to rmarkdown 2.2
\floatplacement{figure}{H}
\usepackage[font=tiny]{subfig} %beamer用

%% 図表タイトルの文字の大きさの設定
%\usepackage[font=small,labelfont={bf,sf}]{caption} %論文用
%\usepackage[font=tiny,labelfont={bf,sf}]{caption} %論文用
%beamerでのcaptionフォントサイズ変更は\setbeamerfont{caption}{size=\tiny}で行う

%% 図表番号の前に来るFigure/Table表記をFig./Tab.表記に変更
%\addto\captionsenglish{\renewcommand{\figurename}{Fig.}}
%\addto\captionsenglish{\renewcommand{\tablename}{Tab.}}

%% 数式の整形
\usepackage{ascmac}

%% 囲み記事の設定
\usepackage{tcolorbox}
\tcbuselibrary{breakable}        

%---------------------------------------------------
%
% フォント指定
%
%---------------------------------------------------

\expandafter\def\csname ver@unicode-math.sty\endcsname{}

\usepackage{fontspec}

%% 欧文・主フォント(セリフ体[ローマン体])
\setmainfont[Scale = 1.1]{Linux Libertine O}

%% 欧文・サンセリフ体フォント(いわゆるゴシック体に見えるフォント)
\setsansfont[Scale = 1.1,%1.00375
             BoldFont = Roboto-Bold.otf,
             ItalicFont = Roboto-italic.otf,
             BoldItalicFont = Roboto-bolditalic.otf]{Roboto-Regular.otf}

%% 欧文・タイプライタ体(コードの表示に使用)
\setmonofont[Scale=MatchLowercase]{zcoN}

%% 数字
\setmathfont[Scale=MatchUppercase]{LibertinusMath-Regular.otf}

%% 和文フォント
\usepackage{zxjatype}

%% 欧文フォントに対する日本語フォントの大きさ(倍率)
\setjafontscale{1} %インストール案内

%% 和文・主フォント(明朝体)
\setjamainfont[BoldFont = SourceHanSansJP-Bold.otf]{SourceHanSerifJP-Light.otf}

%% 和文・ゴシック体
\setjasansfont[Scale=1, BoldFont = SourceHanSansJP-Bold.otf]{SourceHanSansJP-Normal.otf}

%% 和文・タイプライタ体
\setjamonofont{SourceHanSansJP-Normal.otf}

%---------------------------------------------------
%
% natbibを使う際に参考文献フォントを小さくする指定
%
%---------------------------------------------------

%https://tex.stackexchange.com/questions/354101/using-natbib-make-new-entries-not-start-at-new-line

%\def\bibfont{\footnotesize}
%\def\bibfont{\tiny}

%---------------------------------------------------
%
% 行頭インデント・行間の設定
%
%---------------------------------------------------

%% 段落の頭を字下げで始める
\usepackage{indentfirst}

%% 段落の頭の字下げ幅
\setlength{\parindent}{12pt} %日本語用
%\setlength{\parindent}{1.27cm} %英語用 APA 6版
\parskip=0pt

%% 行間の幅
\renewcommand{\baselinestretch}{1.2}

\usepackage{ragged2e}

\usepackage{calc}

%---------------------------------------------------
%
% 言語学関係の設定
%
%---------------------------------------------------

\usepackage{lingmacros}

%% グロス
\makeatletter % {gb4e}packageの妨害コマンドを抑止する
\def\new@fontshape{} % {gb4e}packageの妨害コマンドを抑止する
\makeatother % {gb4e}packageの妨害コマンドを抑止する
\let\mathexp=\exp % save the current (math) definition of \exp
\usepackage{gb4e}\noautomath
\let\gbexp=\exp % save the current (gb4e) definition of \exp
\DeclareRobustCommand{\exp}{\ifmmode\mathexp\else\expandafter\gbexp\fi}

% 1行目(原文)をイタリック体で表記
\let\eachwordone=\itshape

%% 国際音声字母
\usepackage{tipa}
\DeclareTextFontCommand{\textipa}{%
  \fontfamily{ptm}\tipaencoding
}
\renewenvironment{IPA}{\fontfamily{ptm}\tipaencoding}{}


%% 木構造
\usepackage[linguistics]{forest}
%\usepackage{tikz}
%\usepackage{tikz-qtree}

%% HPSGでのAVM(attribute-value matrices)
\usepackage{avm}

\usepackage{fvextra}
\DefineVerbatimEnvironment{Highlighting}{Verbatim}{breaklines,commandchars=\\\{\}, fontsize=\small}

\usepackage{metalogo}

